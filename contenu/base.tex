\section{Quelques commandes de bases}

\subsection{Structuration du document}

\begin{slide}
  \begin{itemize}
    \item 7~niveaux de titres
      \begin{itemize}
	\item \verb.\part.
	\item \verb.\chapter.
	\item \verb.\section. ; \verb.\subsection. ; \verb.\subsection.
	\item \verb.\paragraph. ; \verb.\subparagraph.
      \end{itemize}
    \item Table des matières générale ou par chapitres.
    \item Stabilité dans la numérotation.
    \item Nombreux modèles de numérotation possibles : chiffres romains, arabes, grecs, hébreux, lettres etc.
  \end{itemize}
\end{slide}

\subsection{Éléments non textuels}

\begin{slide}
  \begin{itemize}
    \item Tableaux, images, graphiques.
    \item Notions de flottants pour assurer un positionnement optimum.
  \end{itemize}
\end{slide}

\begin{slide}
  \begin{minted}{tex}
    \begin{figure}[placement]
      \inputgraphics{fichier}
      \caption{Légende}
    \end{figure}
  \end{minted}
\end{slide}

\begin{slide}
  \begin{itemize}
    \item Préciser où doit s'insérer idéalement un flottant:
      \begin{itemize}
	\item À l'emplacement du flottant.
	\item En haut d'une page.
	\item En bas d'une page.
	\item Sur une page dédiée.
	\item (À la fin)
      \end{itemize}
    \item Création de table des flottants.
    \item Autre possibilités : sous-flottants ; nouveau flottants ; flottants orientés dans un autre sens etc.
  \end{itemize}
\end{slide}
