\usepackage{minted}
\usepackage[nobiblatex]{beamer-standrews}
\usepackage{morewrites}
\setmainlanguage{french}
\usepackage{eledmac,eledpar}
\noendnotes
\noeledsec
\usepackage{xifthen}
\usepackage{graphicx,xspace,bidi}
\usepackage{biblatex-philologue}
\bibliography{contenu/intro-latex-shs.bib}
% inspiré de xetex-bidi
\def\reflect#1{{\setbox0=\hbox{#1}\rlap{\kern0.5\wd0
  \special{x:gsave}\special{x:scale -1 1}}\box0 \special{x:grestore}}}
\def\logoXeLaTeX{{\leavevmode$\smash{\hbox{(X\lower.5ex
  \hbox{\kern-.125em\reflect{E})\,}\kern-.1667em \LaTeX}}$}\xspace}

\author{Maïeul Rouquette}
\date{10 février 2015}
\title{Une (trop) brève introduction à \logoXeLaTeX}
\beamerdefaultoverlayspecification{<+->}
\graphicspath{{img/}}

% Quelques commandes utiles
\def\hauteurimg{\dimexpr\textheight-4\baselineskip\relax}

\newcommand{\inputl}[3]{\inputminted[firstline=#2,lastline=#3]{latex}{examples/#1}}

